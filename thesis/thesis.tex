%% Load document class fithesis2
%% {10pt, 11pt, 12pt}
%% {draft, final}
%% {oneside, twoside}
%% {onecolumn, twocolumn}
\documentclass[12pt,final,oneside]{fithesis2}

%% Basic packages
\usepackage[english]{babel}
%\usepackage{cmap}
%\usepackage[T1]{fontenc}
%\usepackage{lmodern}
\usepackage[utf8]{inputenc}
\usepackage{graphicx}

%% Additional packages for colors, advanced
%% formatting options, etc.
%\usepackage{color}
%\usepackage{microtype}
%\usepackage{url}
%\usepackage{cslatexquotes}
%\usepackage{fancyvrb}
%\usepackage[small,bf]{caption}
\usepackage[plainpages=false,pdfpagelabels,unicode]{hyperref}
%\usepackage[all]{hypcap}

%% Fix long URLs in DVIs
%\usepackage{ifpdf}

%\ifpdf
%\else
%  \usepackage{breakurl}
%\fi

%% Packages used to generate various lists
%\usepackage{makeidx}
%\makeindex

%\usepackage[xindy]{glossaries}
%\makeglossary

\usepackage{xcolor}
\newcommand\todo[1]{\textcolor{red}{#1}}

\newcommand\emptypage{\newpage\null\thispagestyle{empty}\newpage}

%% Use STAR and CIRCLE signs for nested
%% itemized lists
%\renewcommand{\labelitemii}{$\star$}
%\renewcommand{\labelitemiii}{$\circ$}

%% Title page information
\thesistitle{String abstract domains}
\thesissubtitle{Master's thesis}
\thesisstudent{Bc. Matej Šuta}
\thesiswoman{false} %% Important when using Slovak or Czech lang
\thesisfaculty{fi}  %% {fi, eco, law, sci, fsps, phil, ped, med, fss}
\thesislang{en}     %% {en, sk, cs}
\thesisyear{Spring 2013}
\thesisadvisor{Mgr. Karel Klíč}

%% Beginning of the document
\begin{document}

%% Front page with a logo and basic thesis information
\FrontMatter
\ThesisTitlePage

\emptypage

%% Thesis declaration (required)
\begin{ThesisDeclaration}
  \DeclarationText
  \AdvisorName
\end{ThesisDeclaration}

%% Thanks (optional)
\begin{ThesisThanks}
My thanks go to ... 
\end{ThesisThanks}

%% Abstract (required)
\begin{ThesisAbstract}
This thesis is about ...
\end{ThesisAbstract}

%% Keywords (required)
\begin{ThesisKeyWords}
GitHub, Thesis, Key, Words, Specific, ...
\end{ThesisKeyWords}

%% Beginning of the thesis itself
\MainMatter

%% TOC (required)
\tableofcontents

%% Thesis text structured using
%% chapters, sections, subsections, etc.
\chapter{Introduction}

As computers get more powerful and complex so does the software. However,
the increasing complexity of software leads to increased number of errors
found in it. There are many techniques that try to prevent software bugs,
e.g.~testing, formal verification, etc. Abstract interpretation is the
approach that is the subject of this thesis. 
One can focus on many aspect of a program to analyze. One of them
is how program manipulates strings. A string is very flexible data
structure that allows programmer to represent practically anything.
In fact, most of the programs written today deal with strings somehow.
They may be used for program input/output or to generate whole other
programs (SQL). Thus, it's important to know how programs treat strings.
\todo{this needs some work}

This thesis describes string abstract domains in context of abstract
interpretation framework. It is based on papers published
by Cousot~\cite{CousotCousot77-1} and other authors. The goal of the thesis
is to design, implement and integrate several abstract domains for strings
into \texttt{canal} abstract interpreter. \texttt{canal} is a tool designed
to analyze behavior of application programs written in \texttt{C}.

The text has the following outline: chapter~\ref{chap:preliminaries}
presents preliminaries required to understand the abstract interpretation.
In chapter~\ref{chap:design} several string abstract domains are designed.
The implementation of designed domains is described in great detail in
chapter~\ref{chap:implementation}. Finally, thesis findings are summarized
in chapter~\ref{chap:conclusion}. Appendix~\ref{chap:instructions} contains
instructions for compiling and running the \texttt{canal} abstract
interpreter.


\chapter{Abstract interpretation and string abstract domains}
\label{chap:preliminaries}

intro

static analysis - term definition, purpose, example

halting problem - definition, example, relation to static analysis

types of static analysis, formal, etc, short description of every type

other ways of evaluating SW quality

abstract interpretation - term definition, purpose, example


\chapter{Solution design}
\label{chap:design}


\chapter{Implementation}
\label{chap:implementation}


\chapter{Conclusion}
\label{chap:conclusion}


%% Lists of tables and figures, glossary, etc.
%\printindex
%\printglossary
%\listoffigures
%\listoftables

%% Bibliography
\begin{thebibliography}{99}

\bibitem{CousotCousot77-1}
P{.} Cousot and R{.}Cousot.
\newblock Abstract interpretation: a unified lattice model for static
  analysis of programs by construction or approximation of fixpoints.
\newblock In \emph{Conference Record of the Fourth Annual ACM
  SIGPLAN-SIGACT Symposium on Principles of Programming Languages},
  pages 238--252, Los Angeles, California, 1977. ACM Press, New York,
  NY, USA.

\bibitem{Cousot00-1}
P{.} Cousot.
\newblock Abstract interpretation based formal methods and future
  challenges.
\newblock In \emph{Informatics - 10 Years Back, 10 Years Ahead},
  volume 2000 of LNCS, pages 138--156. Springer, 2000.

\end{thebibliography}


%% Additional materials
\appendix

\chapter{Instructions for running \texttt{canal}}
\label{chap:instructions}

%% End of the whole document
\end{document}

